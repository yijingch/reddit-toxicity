\documentclass{article}



\usepackage{arxiv}

\usepackage[utf8]{inputenc} % allow utf-8 input
\usepackage[T1]{fontenc}    % use 8-bit T1 fonts
\usepackage{hyperref}       % hyperlinks
\usepackage{url}            % simple URL typesetting
\usepackage{booktabs}       % professional-quality tables
\usepackage{amsfonts}       % blackboard math symbols
\usepackage{nicefrac}       % compact symbols for 1/2, etc.
\usepackage{microtype}      % microtypography
\usepackage{lipsum}		% Can be removed after putting your text content
\usepackage{graphicx}
\usepackage{natbib}
\usepackage{doi}



\title{A template for the \emph{arxiv} style}

%\date{September 9, 1985}	% Here you can change the date presented in the paper title
%\date{} 					% Or removing it

\author{ \href{https://orcid.org/0000-0002-9311-6357}{\includegraphics[scale=0.06]{orcid.pdf}\hspace{1mm}Yijing Chen}\\
%		\thanks{Use footnote for providing further
%		information about author (webpage, alternative
%		address)---\emph{not} for acknowledging funding agencies.} \\
	Department of Network and Data Science\\
	Central European University\\
	Wien, Austria 1100 \\
	\texttt{chen\_yijing@phd.ceu.edu} \\
	%% examples of more authors
	%% \AND
	%% Coauthor \\
	%% Affiliation \\
	%% Address \\
	%% \texttt{email} \\
	%% \And
	%% Coauthor \\
	%% Affiliation \\
	%% Address \\
	%% \texttt{email} \\
	%% \And
	%% Coauthor \\
	%% Affiliation \\
	%% Address \\
	%% \texttt{email} \\
}

% Uncomment to remove the date
%\date{}

% Uncomment to override  the `A preprint' in the header
%\renewcommand{\headeright}{Technical Report}
\renewcommand{\undertitle}{DNDS 6012: Social Network Fall 2021}
%\renewcommand{\shorttitle}{\textit{arXiv} Template}

%%% Add PDF metadata to help others organize their library
%%% Once the PDF is generated, you can check the metadata with
%%% $ pdfinfo template.pdf
\hypersetup{
pdftitle={A template for the arxiv style},
pdfsubject={q-bio.NC, q-bio.QM},
pdfauthor={Yijing Chen},
pdfkeywords={social networks, social media, online community},
}

\begin{document}
\maketitle

\begin{abstract}
	\lipsum[1]
\end{abstract}



% keywords can be removed
\keywords{First keyword \and Second keyword \and More}


\section{Introduction}

Despite their contributions in facilitating political communication and potentially promoting deliberative democracy, social network sites are increasingly accused of breeding or amplifying toxic behaviors such as spreading mis/disinformation and hate speech, which pressures the platforms to be more critical about their ``free speech'' rhetoric and moderate user-generated content with extra caution. For instance, Reddit has been banning controversial subreddits that violate its content policies, including {\tt r/The\_Donald}, one of the most active political subreddits with more than 790,000 subscribers. Existing works have explored the impact of such interventions at both the user- and the community-level\cite{copland2020reddit, chandrasekharan2017you}, but mostly adopt an isolated approach that analyzes individual data points without considering the underlying dynamics in the network structure. Thus, their implications on the efficacy of content moderation are still limited, not answering whether and how these interconnected elements (i.e., users and communities) maintain their interactions after the moderation. Therefore, my project hopes to observe how subreddit bans introduce topological changes of user and community network in online political discussions. More specifically, I plan to retrieve comments and submissions from 65 political subreddits\cite{rajadesingan2020quick}, build projection networks of the user-community bipartite on both side, and compare the network characteristics before and after the subreddit bans.


\section{Dataset}

\begin{table}[]
\begin{tabular}{l|cccccc}
Network                   & $\rho_{kx}$ & $r_{xx}$ & $H$    & $\langle x \rangle$ &             & $\langle x \rangle_nn$ \\ \hline
Empirical                 & 0.0658**    & -0.0001  & 0.6170 & 0.2674              & \textless{} & 0.2724                 \\
Shuffle toxicity per post & -0.0035     & -0.0001  & 0.6170 & 0.2672              & $\approx$   & 0.2673                
\end{tabular}
\end{table}

\section{Related Works}

\section{Methods}

\section{Results and Discussion}

\section*{Acknowledgements}

%\subsection{Figures}
%\lipsum[10]
%See Figure \ref{fig:fig1}. Here is how you add footnotes. \footnote{Sample of the first footnote.}
%\lipsum[11]

%\begin{figure}
%	\centering
%	\fbox{\rule[-.5cm]{4cm}{4cm} \rule[-.5cm]{4cm}{0cm}}
%	\caption{Sample figure caption.}
%	\label{fig:fig1}
%\end{figure}
%
%\subsection{Tables}
%See awesome Table~\ref{tab:table}.
%
%The documentation for \verb+booktabs+ (`Publication quality tables in LaTeX') is available from:
%\begin{center}
%	\url{https://www.ctan.org/pkg/booktabs}
%\end{center}


%\begin{table}
%	\caption{Sample table title}
%	\centering
%	\begin{tabular}{lll}
%		\toprule
%		\multicolumn{2}{c}{Part}                   \\
%		\cmidrule(r){1-2}
%		Name     & Description     & Size ($\mu$m) \\
%		\midrule
%		Dendrite & Input terminal  & $\sim$100     \\
%		Axon     & Output terminal & $\sim$10      \\
%		Soma     & Cell body       & up to $10^6$  \\
%		\bottomrule
%	\end{tabular}
%	\label{tab:table}
%\end{table}



\bibliographystyle{plain}
\bibliography{references}  


\end{document}
